\documentclass[12pt,a4paper]{article}
\usepackage[margin=1.2in]{geometry}
\usepackage{graphicx}
\usepackage{amsmath}
\usepackage{framed, color}
\definecolor{shadecolor}{rgb}{0.9, 0.9, 0.9}
\setlength{\topmargin}{0cm}

% Create friendly environments for theorems, propositions, &c.
\begin{document}
\section{Nota sobre a natureza das probabilidades}
Para poder discorrer sobre a estimação de incertezas e a atribuição de probabilidades a diferentes resultados de um modelo,
é necessário examinar com atenção o que entendemos sobre probabilidade.

A teoria matemática da probabilidade remonta aos séculos XVI e XVII, nos quais Girolamo Cardano, Galileu Galilei, Blaise Pascal
e Pierre Fermat desenvolveram métodos para resolver problemas envolvendo combinações de resultados em jogos de dados e outros
jogos de azar. 
Essa origem, juntamente com a facilidade e a universalidade de problemas envolvendo jogos de azar, explicam porque tantos textos
introdutórios sobre estatística empregam exemplos envolvendo rolar dados e tirar cartas de baralhos. Embora a simplicidade
destes exemplos ajude a transmitir os conceitos de probabilidade com facilidade, isso causa muitas vezes a sensação de que
as regras da probabilidade servem apenas para os casos simples, e que estas não se aplicam, ou não se adequam totalmente aos 
problemas de natureza complexa com os quais a biologia e a ecologia têm de enfrentar. De fato, é muito difícil justificar que
questões como ``qual a chance de que uma população de jararaca-ilhoa entre em extinção nos próximos dez anos?'' sejam embasadas
nas mesmas idéias que ``qual a chance de uma moeda jogada para cima dar coroa?''. É mais natural tratar a primeira pergunta como
um questionamento sobre o nosso conhecimento atual dos processos biológicos e da condição de vida dos indivíduos dessa espécie
do que um questionamento sobre um processo físico simples que pode ser repetido um grande número de vezes. 

Porém, mesmo textos escritos por estatísticos empregam frequentemente definições conflitantes sobre probabilidades: às vezes,
elas são usadas para representar algo pessoal e subjetivo, como um nível de confiança; outras vezes são apresentadas como a
razão entre diferentes contagens. Vamos delinear aqui algumas das principais interpretações possíveis, juntamente com
algumas críticas a todas elas.

books.google.com.br/books?id=es0AAAAAcAAJ preface, x



\section{O papel da incerteza no método científico}
John Venn, no prefácio do seu estudo de 1866, escreve:

``The science of Probability occupies at present a somewhat anomalous position. It is impossible, I think, not to observe
in it some of the marks and consequent disadvantages of a {\em sectional} study. By a small body of ardent students it
has been cultivated with great assiduity, and the results they have obtained will always be reckoned among the most 
extraordinary products of mathematical genius. But by the general body of thinking men its principles seem to be regarded
with indifference or suspicion. Such persons may admire the ingenuity displayed, and be struck with the profoundity
of many of the calculatins, but there seems to them, if I may so express it, an {\em unreality} about the whole treatment
of the subject. To many persons the mention of Probability suggests little else than the notion of a set of rules, very
ingenious and profound rules no doubt, with which mathematicians amuse themselves by setting and solving puzzles.''

No século e meio que se passou desde esta publicação, a aplicação da estatística se tornou uma obrigatoriedade no meio científico.
Em qualquer periódico renomado da área, é virtualmente impossível publicar um artigo experimental que não mencione alguma propriedade
estatística a respeito das amostras coletadas. Por muitos anos, a estatística vigente nas análises biológicas teve uma
forte inspiração frequentista, com testes de hipóteses e valores {\em p} sendo 
No entanto, existe uma ampla crítica à forma como essa estatística é
utilizada nas ciências biológicas e na área de medicina \cite{Ioannidis05}.
Muito da crítica decorre da interpretação
simplista assumida por muitos cientistas a respeito do valor {\em p} reportado sobre um problema, mas as suas raízes devem
ser traçadas mais profundamente em uma falta de consciência dos cientistas sobre as interpretações a respeito da natureza
das probabilidades que são pressupostas por uma escola de pensamento, mas raramente examinadas profundamente. 
De outro lado, análises recentes sugerem que o tamanho dos efeitos em várias áreas da ecologia são tão pequenos que estes
dificilmente serão corretamente detectados pelos procedimentos tradicionais \cite{Jennions03}. Em áreas como a ecologia do
comportamento, o poder dos testes estatísticos é muito pequeno, e esse problema é acentuado quando são utilizados métodos
(como o método de Bonferroni) para corrigir o valor {\em p} reportado.

Novamente, citamos John Venn:
``Students of Philosophy in general have thence conceived a prejudice against Probability, which has for the most part
deterred them from examining it. As soon as a subject comes to be considered `mathematical' its claims seem generally,
by the mass of readers, to be either on the one hand scouted or at least courteously rejected, or on the other hand
to be blindly accepted with all their assumed consequences. Of impartial and liberal criticism it obtains little or nothing.''


``There comes a time in the life of a scientist when he must convince himself
either that his subject is so robust from a statistical point of view that
the finer points of statistical inference are irrelevant, or that the precise
mode of inference he adopts is satisfactory. Most will be able to settle for
the former, and they are perhaps fortunate in being able to conserve their
intellectual energy for their main interests; but some will be forced, by the
paucity of their data or the complexity of their inferencies, to examine
the finer points of their own arguments, and in so doing they are likely
to become lost in the quicksands of the debate on statistical inference.''
(A.W.F. Edwards, Likelihood)

\end{document}
