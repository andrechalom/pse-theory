\documentclass[12pt,a4paper]{article}
\usepackage[margin=1.2in]{geometry}
\usepackage{graphicx}
\usepackage[utf8]{inputenc}
\usepackage{amsmath}
\usepackage{framed, color}
\definecolor{shadecolor}{rgb}{0.9, 0.9, 0.9}
\setlength{\topmargin}{0cm}

% Create friendly environments for theorems, propositions, &c.
\begin{document}
\title{Sobre probabilidade e inferência}
\maketitle

``There comes a time in the life of a scientist when he must convince himself
either that his subject is so robust from a statistical point of view that
the finer points of statistical inference are irrelevant, or that the precise
mode of inference he adopts is satisfactory. Most will be able to settle for
the former, and they are perhaps fortunate in being able to conserve their
intellectual energy for their main interests; but some will be forced, by the
paucity of their data or the complexity of their inferencies, to examine
the finer points of their own arguments, and in so doing they are likely
to become lost in the quicksands of the debate on statistical inference.''
(A.W.F. Edwards, Likelihood)

\section{Conceitos históricos sobre a natureza das probabilidades}
Para poder discorrer sobre a estimação de incertezas e a atribuição de probabilidades a diferentes resultados de um modelo,
é necessário examinar com atenção o que entendemos sobre probabilidade.

A teoria matemática da probabilidade remonta aos séculos XVI e XVII, nos quais Girolamo Cardano, Galileu Galilei, Blaise Pascal
e Pierre Fermat desenvolveram métodos para resolver problemas envolvendo combinações de resultados em jogos de dados e outros
jogos de azar. 
Essa origem, juntamente com a facilidade e a universalidade de problemas envolvendo jogos de azar, explicam porque tantos textos
introdutórios sobre estatística empregam exemplos envolvendo rolar dados e tirar cartas de baralhos. Embora a simplicidade
destes exemplos ajude a transmitir os conceitos de probabilidade com facilidade, isso causa muitas vezes a sensação de que
as regras da probabilidade servem apenas para os casos simples, e que estas não se aplicam, ou não se adequam totalmente aos 
problemas de natureza complexa com os quais a biologia e a ecologia têm de enfrentar. De fato, é muito difícil justificar que
questões como ``qual a chance de que uma população de jararaca-ilhoa entre em extinção nos próximos dez anos?'' sejam embasadas
nas mesmas idéias que ``qual a chance de uma moeda jogada para cima dar coroa?''. É mais natural tratar a primeira pergunta como
um questionamento sobre o nosso conhecimento atual dos processos biológicos e da condição de vida dos indivíduos dessa espécie
do que um questionamento sobre um processo físico simples que pode ser repetido um grande número de vezes. 

Porém, mesmo textos escritos por estatísticos empregam frequentemente definições conflitantes sobre probabilidades: às vezes,
elas são usadas para representar algo pessoal e subjetivo, como um nível de confiança; outras vezes são apresentadas como a
razão entre diferentes contagens.

A primeira tentativa de definir formalmente o conceito de probabilidade vem em 1814,
com o trabalho de Pierre Simon de Laplace:

``La théorie des hasards consiste à réduire tous les évènemens du même genre, à un certain nombre de cas également
possibles, c'est-à-dire tels que nous soyons également indécis sue leur existence, et à determiner le nombre de cas
favorables à l'évènement dont on cherche la probabilité. Le rapport de ce nombre à celui de tous les cas possibles,
est la mesure de cette probabilité'' \cite{Laplace1814}

(``A teoria da probabilidade consiste em reduzir todos os eventos de um mesmo tipo a um certo número de casos igualmente
possíveis, isso é, tal que sejamos igualmente indecisos sobre a sua ocorrência, e a determinar o número de casos
favoráveis ao evento ao qual buscamos a probabilidade. A razão deste número para o número de todos os casos possíveis
é a medida desta probabilidade'')

Para Laplace, todos os eventos que presenciamos seriam resultados de leis físicas
imutáveis, e uma inteligência superior, dotada do conhecimento do estado do universo em um dado instante, poderia prever
todos os eventos futuros sem qualquer incerteza. No entanto, nosso conhecimento limitado, tanto do estado do universo
quanto das leis que o regem, faz com que não possamos fazer previsões para um grande número de sistemas. O estudo das 
probabilidades se coloca, então, como um apoio ao nosso poder de realizar previsões sobre o universo, 
enquanto não possuímos o conhecimento das leis e estados necessária para realizar previsões certeiras. 

Uma dificuldade com essa definição de Laplace é que a demarcação de ``eventos igualmente prováveis'' deve ser feita
com base em argumentos que não partam da idéia de probabilidade; caso contrário, podemos incorrer em raciocínios
circulares. Por exemplo, se uma moeda tem 2 lados perfeitamente simétricos, e a lançamos de forma a não privilegiar um dos lados,
podemos dizer que o número de eventos totais é dois, e o número de eventos favoráveis a tirar coroa é um. Não há qualquer motivo
para supor que um dos lados seja mais propenso a cair para cima que o outro, visto a moeda ser perfeitamente simétrica; logo,
a probabilidade de tirar coroa em um lançamento é de $\frac{1}{2}$. Aqui, o ato de atribuir iguais chances aos dois
lados da moeda vem de um argumento físico de simetria. Isso é dizer que, para Laplace,
a ciência da probabilidade deve se basear em leis físicas do mundo natural, e não no nosso estado presente de conhecimento 
sobre o mundo. Esta interpretação faz com que as leis da probabilidade, estritamente, só possam ser aplicadas para 
sistemas simples, sobre os quais temos um alto conhecimento e controle. Laplace, no entanto, não parece ter levado essa 
definição à suas últimas consequências, já que ele discute no tratado citado acima problemas referentes a fazer inferências 
sobre testemunhos judiciais, nos 
quais a testemunha poderia mentir, com uma certa probabilidade, ou ter se equivocado, com outra probabilidade. No entanto, 
nenhuma explicação convincente é dada sobre como medir essas probabilidades.

Uma visão alternativa é dada por Augustus de Morgan, conhecido principalmente por suas leis em lógica proposicional, 
no seu tratado Formal Logic, de 1847. Para de Morgan, a única certeza que podemos ter é a de nossa própria existência. 
Este conhecimento
não é passível de ser refutado. No entanto, qualquer outra proposição feita deve ser acompanhada por 
um {\em grau de conhecimento} subjetivo:

``It will be found that, frame what circunstances we may, we cannot invent a case of purely objective probability.
I put ten white balls and ten black ones into an urn, and lock the door of the room. I may feel well assured that,
when I unlock the room again, and draw a ball, I am justified in saying it is an even chance that it will be a white one.
If all the metaphysicians who ever wrote on probability were to witness the trial, they would, each in his own sense and 
manner, hold me right in my assertion. But how many things there are to be taken for granted! Do my eyes still distinguish
colours as before? Some persons never do, and eyes alter with age. Has the black paint melted, and blackened the white balls?
Has any one else possessed a key of the room, or got in at the window, and changed the balls? We may be {\em very sure},
as those words are commonly used, that none of these things have happened, and it may turn out (and I have no doubt will do so,
if the reader try the circumstances) that the ten white and ten black balls will be found, as distinguishable as ever, and
unchanged. But for all that, there is much to be assumed in reckoning upon such a result, which is not so objective (in the
sense in which I used the word) as the knowledge of what the balls were when they were put into the urn.'' \cite{deMorgan1847}

(``Chegaremos à conclusão de que, mesmo enquadrando as circustâncias que quisermos, não podemos inventar nenhum caso onde a 
probabilidade é puramente objetiva. [Suponha que] eu coloque dez bolas brancas e dez bolas pretas em uma urna, e tranque a porta
da sala. Posso me sentir seguro de que, quando eu destrancar a sala, estarei justificado em dizer que há uma chance igual de 
tirar uma bola branca ou preta. Se todos os estudantes de metafísica que já escreveram sobre probabilidades fossem testemunhas
desse experimento, cada um deles, à sua maneira, concordar com minha afirmação. Mas quantas coisas estamos tomando como 
garantidas! Será que meus olhos ainda distinguem as cores como antes? Algumas pessoas nunca as distinguem, e os olhos se 
alteram com a idade. Será que a tinta preta derreteu, e escureceu as bolas brancas? Será que alguém mais tinha uma chave da sala,
ou entrou pela janela, e trocou as bolas? Podemos estar {\em muito seguros}, como se diz comumente, de que nada disso aconteceu,
e pode ser que ocorra (como eu não duvido que será o caso, se o leitor tentar o experimento) que as dez bolas brancas e as 
dez bolas pretas vão estar, bem distinguíveis como antes, e inalteradas. Mas até chegar nisso, precisamos presumir um grande
número de fatos, que não tem o caráter objetivo (no sentido em que eu usei a palavra) que tem o conhecimento de como as bolas
eram quando eu as coloquei na urna'').

A teoria de probabilidades segundo de Morgan, portanto, lida com graus de conhecimento subjetivos. Assim, ao perguntar para
uma pessoa comum qual é a chance de que um lançamento de moeda resulte em cara, essa pessoa pode responder $\frac{1}{2}$, e 
esta é a medida de probabilidade correta, dada a informação que ele possui sobre o problema. Uma outra pessoa, que sabe
que esta moeda é viciada, ou que é capaz de jogar a moeda de forma a privilegiar um resultado, pode dar outra resposta
completamente diferente, e ainda assim estará correta. Para de Morgan, o passo essencial na construção de uma probabilidade
consiste em {\em medir} o grau de certeza que temos em uma proposição. *** TALVEZ ELABORAR AQUI? ***

A próxima crítica à visão de Laplace vem de John Venn, em seu trabalho de 1866. *** DEFINICAO ***

Há um problema claro com essa visão, que é o fato de que não temos acesso a infinitas experiências para determinar o valor
de uma dada probabilidade. Mesmo quando temos acesso a um grande conjunto de dados para estimar o valor de uma determinada
probabilidade, esse conjunto pode ser suficientemente heterogêneo para que nossa conclusão se torne incorreta. Como escreve Venn,

``At the present time the average duration of life in England may be, say, forty years; but a century ago it was decidedly less;
several centuries ago it was presumably very much less (...). Let us assume that the regularity is fixed and permanent. It is
making a hypothesis which may not be altogether consistent with fact, but which is forced upon us for the purpose of securing
precision of statement and definition.''\cite{Venn1866}

*** CASO PATOLOGICO de inverter o resultado ***
Embora a definição de experimento possa ser facilmente reformulada para excluir esses casos visivelmente patológicos, a análise
deles demonstra que a definição de probabilidade dada pelo paradigma frequentista não pode ser aplicada para {\em todo e qualquer}
problema envolvendo a determinação de uma probabilidade.

**** Discutir Venn e os frequentistas ****
%The Logic of Chance:
%http://archive.org/stream/logicofchance029416mbp#page/n39/mode/2up página 41

%http://en.wikipedia.org/wiki/Probability_interpretations
**** Discutir a "resposta" dos Bayesianos para o paradigma frequentista ****

**** Lindley's paradox? ****
%http://en.wikipedia.org/wiki/Jeffreys%E2%80%93Lindley_paradox

%Livro do Edwards
**** Discutir o livro do Edwards e verossimilhança ****

\section{O papel da incerteza na produção do conhecimento científico}
**** discutir a visão de Popper ****

John Venn, no prefácio do seu estudo de 1866, escreve:

``The science of Probability occupies at present a somewhat anomalous position. It is impossible, I think, not to observe
in it some of the marks and consequent disadvantages of a {\em sectional} study. By a small body of ardent students it
has been cultivated with great assiduity, and the results they have obtained will always be reckoned among the most 
extraordinary products of mathematical genius. But by the general body of thinking men its principles seem to be regarded
with indifference or suspicion. Such persons may admire the ingenuity displayed, and be struck with the profoundity
of many of the calculatins, but there seems to them, if I may so express it, an {\em unreality} about the whole treatment
of the subject. To many persons the mention of Probability suggests little else than the notion of a set of rules, very
ingenious and profound rules no doubt, with which mathematicians amuse themselves by setting and solving puzzles.''
\cite{Venn1866}

No século e meio que se passou desde esta publicação, a aplicação da estatística se tornou uma obrigatoriedade no meio científico.
Em qualquer periódico renomado da área, é virtualmente impossível publicar um artigo experimental que não mencione alguma propriedade
estatística a respeito das amostras coletadas. Por muitos anos, a estatística vigente nas análises biológicas teve uma
forte inspiração frequentista, com testes de hipóteses e valores {\em p} sendo requeridos para publicações. 
No entanto, existe uma ampla crítica à forma como essa estatística é
utilizada nas ciências biológicas e na área de medicina \cite{Ioannidis05}.
Muito da crítica decorre da interpretação
simplista assumida por muitos cientistas a respeito do valor {\em p} reportado sobre um problema, mas as suas raízes devem
ser traçadas mais profundamente em uma falta de consciência dos cientistas sobre as interpretações a respeito da natureza
das probabilidades que são pressupostas por uma escola de pensamento, mas raramente examinadas profundamente. 
De outro lado, análises recentes sugerem que o tamanho dos efeitos em várias áreas da ecologia são tão pequenos que estes
dificilmente serão corretamente detectados pelos procedimentos tradicionais \cite{Jennions03}. Em áreas como a ecologia do
comportamento, o poder dos testes estatísticos é muito pequeno, e esse problema é acentuado quando são utilizados métodos
(como o método de Bonferroni) para corrigir o valor {\em p} reportado.

Novamente, citamos John Venn:

``Students of Philosophy in general have thence conceived a prejudice against Probability, which has for the most part
deterred them from examining it. As soon as a subject comes to be considered `mathematical' its claims seem generally,
by the mass of readers, to be either on the one hand scouted or at least courteously rejected, or on the other hand
to be blindly accepted with all their assumed consequences. Of impartial and liberal criticism it obtains little or nothing.''
\cite{Venn1866}


\bibliographystyle{plain}
\bibliography{chalom}

\end{document}
