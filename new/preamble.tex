\documentclass[12pt,a4paper]{article}
\usepackage[margin=1.2in]{geometry}
\usepackage{graphicx}
\usepackage{amsmath}
\usepackage{framed, color}
\definecolor{shadecolor}{rgb}{0.9, 0.9, 0.9}
\setlength{\topmargin}{0cm}

% Create friendly environments for theorems, propositions, &c.
\begin{document}

``In an ideal world, all data would come from well-designed experiments
and would be sufficient to simultaneously estimate all parameters using rigorous
statistical procedures. The world is not ideal. One must often combine estimates
from different experiments, or supplement high-quality data (...) with uncertain
data, or even assumptions (...). Think carefully about whether your conclusions
may be artifacts of your assumptions or calculations, and document those assumptions
and calculations so that your reader can ask the same question, and then carry on.''
(Hal Caswell, Matrix Population Models)

``There comes a time in the life of a scientist when he must convince himself
either that his subject is so robust from a statistical point of view that
the finer points of statistical inference are irrelevant, or that the precise
mode of inference he adopts is satisfactory. Most will be able to settle for
the former, and they are perhaps fortunate in being able to conserve their
intellectual energy for their main interests; but some will be forced, by the
paucity of their data or the complexity of their inferencies, to examine
the finer points of their own arguments, and in so doing they are likely
to become lost in the quicksands of the debate on statistical inference.''
(A.W.F Edwards, Likelihood)

%``How often have I said to you that when you have eliminated the impossible,
%whatever remains, {\em however improbable}, must be the truth?''
%(Sherlock Holmes {\em in} A. Conan Doyle, The Sign of the Four)

``To say that the probability of rain is 60\% is neither to say that
`it will rain', nor that `it will not rain': however, in this way,
everyone is in a better position to act than they would be had
metereologists sharply answered either `yes' or `no'.''
(B. de Finetti, Philosophical Lectures on Probability)

\end{document}
