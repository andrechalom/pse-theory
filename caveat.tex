\section{{\em Caveat} sobre o uso de estatísticas}

Ap\'os finalizar a an\'alise de incerteza de um modelo, pode parecer
tentador utilizar a m\'edia dos resultados do modelo como uma medida
da tend\^encia central dos resultados. No entanto, isso pode levar
a um resultado equivocado, e devemos usar sempre o m\'aximo
(ou os m\'aximos) da densidade dos resultados para representar os
pontos mais prov\'aveis da sua distribui\c c\~ao.

Como exemplo, considere o modelo simples

\begin{equation}
	y = x
\end{equation}

Onde o par\^ametro $x$ \'e estimado, a partir da realiza\c c\~ao de um
processo Poisson, como tendo o valor $\hat x$. Para estudar a incerteza
do resultado do modelo $y$, vamos utilizar a fun\c c\~ao de verossimilhan\c ca
de $x$, normalizada de forma a representar uma probabilidade. Lembremos que
a fun\c c\~ao de verossimilhan\c ca de uma distribui\c c\~ao Poisson
\'e:

\begin{equation}
	L \left( \lambda | \hat x \right) = C \lambda^{\hat x} \mathrm{e}^{-\lambda}
\end{equation}

Onde $C$ \'e uma constante multiplicativa que deve ser ajustada de forma
que a integral de $L(\lambda | \hat x)$ seja igual a 1:

\begin {eqnarray*}
\int_0^\infty C \lambda^{\hat x} \mathrm{e}^{-\lambda} & = & 1 \\
C & = & \left( \int_0^\infty  \lambda^{\hat x} \mathrm{e}^{-\lambda} \right)^{-1} \\
C & = & \Gamma(\hat x +1)^{-1}
\end{eqnarray*}

O m\'aximo da fun\c c\~ao $L(\lambda | \hat x)$, descrita acima, ocorre em $\lambda = \hat x$.
Logo, o valor de $\hat x$ deve ser usado como estimador pontual do valor mais prov\'avel
do par\^ametro $x$. Da mesma forma, intervalos de confian\c ca para o par\^ametro $x$ devem
ser constru\'idos ao redor de $\hat x$.

Ap\'os tomar amostras desta distribui\c c\~ao, construimos a distribui\c c\~ao
de resultados:
\begin{equation}
	D(y) = C y^{\hat x} \mathrm{e}^{-y}
\end{equation}

Cuja m\'edia \'e dada por

\begin{eqnarray*}
	<D(y)> & = & \int_0^\infty y C y^{\hat x} \mathrm{e}^{-y} dy \\
	       & = & \int_0^\infty C y^{\hat x+1} \mathrm{e}^{-y} dy \\
		   & = & \Gamma(\hat x +2) C\\
	       & = & \frac{\Gamma(\hat x +2)}{\Gamma(\hat x + 1)} \\
		   & = & \hat x + 1
\end{eqnarray*}

Note ent\~ao que, se escolhemos representar a distribui\c c\~ao dos resultados pela sua m\'edia,
vamos estar usando $\hat x + 1$, enquanto que o valor mais prov\'avel para $y$ \'e $\hat x$, o mesmo
que para o par\^ametro $x$. Da mesma forma, ao construir intervalos de confian\c ca para o resultado
$y$, estes devem ser feitos ao redor de $\hat x$, e n\~ao ao redor da m\'edia $\hat x + 1$. 

Durante a an\'alise de matrizes de Lefkovitch com denso-depend\^encia para
{\em E. edulis}, a popula\c c\~ao final m\'edia \'e de cerca de 17.000 \'arvores (uma estimativa
muito alavancada por uma pequena parcela de simula\c c\~oes com popula\c c\~ao final na casa dos 30 a 40 mil).
J\'a a densidade m\'axima \'e atingida pr\'oxima a 5.000 \'arvores.
