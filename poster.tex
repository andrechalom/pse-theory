\documentclass[final]{beamer}
\mode<presentation> {\usetheme{Dreuw}}
\usepackage{amsmath,amsthm,amssymb,latexsym, graphicx}
\usepackage[english]{babel}
\usepackage[utf8]{inputenc}
\usepackage[orientation=portrait,size=a0,scale=1.4,debug]{beamerposter}

\title{Sensitivity of a density-dependent population matrix model}
\author{Andr\'e Chalom$^1$ \& Paulo In\'acio Prado$^2$}
\institute{$^1$ Programa Interunidades de Pós-Graduação em Bioinformática, 
 $^2$ Instituto de Biociências, Universidade de São Paulo, Brasil.}

  \begin{document}
  \begin{frame}{} 
    \vfill
    \begin{block}{\large Abstract}
In recent years, we are seeing the formulation and use of
elaborate and complex models in ecological studies.
The questions related to the efficient, systematic and error-proof
exploration of parameter spaces are of great importance to better understand,
estimate confidences and make use of the output from these models. 
In this work,
we investigate a structured population growth model
using Latin Hypercube Sampling (LHS) and Extended Fourier Amplitude Sensitivity
Testing (eFAST).
We also contrast our findings with results from previously used techniques, 
known as matrix sensitivity and elasticity analyses.
    \end{block}
    \vfill
    \begin{columns}[t]
      \begin{column}{.48\linewidth}
        \begin{block}{Introduction}
			X \cite{McKay79}
          \begin{itemize}
          \item some items
          \item some items
          \item some items
          \item some items
          \end{itemize}
        \end{block}
    \vfill
        \begin{block}{Methods}
          \begin{itemize}
          \item some items and $\alpha=\gamma, \sum_{i}$
          \item some items
          \item some items
          \item some items
          \end{itemize}
          $$\alpha=\gamma, \sum_{i}$$
        \end{block}
		\vfill
        \begin{block}{Bibliography}
		\bibliographystyle{plain}
		\bibliography{chalom}
        \end{block}
      \end{column}
      \begin{column}{.48\linewidth}

        \begin{block}{PRCC results}
		\begin{figure}
		\begin{center}
		\includegraphics[width=0.8\textwidth]{leslie-PRCC}
		\end{center}
		\caption{Partial Rank Correlation Coefficients for the density independent (a) and density dependent (b) models. The bars are confidence intervals, generated by bootstraping 1000 times}
		\end{figure}
        \end{block}
\vfill
        \begin{block}{eFAST results}
		\begin{figure}
		\begin{center}
		\includegraphics[width=0.8\textwidth]{leslie-fast}
		\end{center}
		\caption{eFAST analysis for the density independent (a) and density dependent (b) models. The bars represent the first and total order estimates for the sensitivity of each parameter in the model output.}
		\end{figure}
        \end{block}
      \end{column}
    \end{columns}
  \end{frame}
\end{document}
