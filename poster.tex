\documentclass[final]{beamer}
\mode<presentation> {\usetheme{Dreuw}}
\usepackage{amsmath,amsthm,amssymb,latexsym, graphicx}
\usepackage[english]{babel}
\usepackage[utf8]{inputenc}
\usepackage[orientation=portrait,size=a0,scale=1.4,debug]{beamerposter}

\title{Sensitivity of a density-dependent population matrix model}
\author{Andr\'e Chalom$^1$ \& Paulo In\'acio Prado$^2$}
\institute{$^1$ Programa Interunidades de Pós-Graduação em Bioinformática, 
 $^2$ Instituto de Biociências, Universidade de São Paulo, Brasil.}

  \begin{document}
  \begin{frame}{} 
    \vfill
    \begin{block}{\large Abstract}
In recent years, we are seeing the formulation and use of
elaborate and complex models in ecological studies.
The questions related to the efficient, systematic and error-proof
exploration of parameter spaces are of great importance to better understand,
estimate confidences and make use of the output from these models. 
In this work,
we investigate a structured population growth model
using Latin Hypercube Sampling (LHS) and Extended Fourier Amplitude Sensitivity
Testing (eFAST).
We also contrast our findings with results from previously used techniques, 
known as matrix sensitivity and elasticity analyses.
    \end{block}
    \vfill
    \begin{columns}[t]
      \begin{column}{.48\linewidth}
        \begin{block}{Introduction}
          \begin{itemize}
          \item some items
          \item some items
          \item some items
          \item some items
          \end{itemize}
        \end{block}
    \vfill
      \end{column}
      \begin{column}{.48\linewidth}
        \begin{block}{Introduction}
          \begin{itemize}
          \item some items and $\alpha=\gamma, \sum_{i}$
          \item some items
          \item some items
          \item some items
          \end{itemize}
          $$\alpha=\gamma, \sum_{i}$$
        \end{block}

        \begin{block}{Introduction}
          \begin{itemize}
          \item some items
          \item some items
          \item some items
          \item some items
          \end{itemize}
        \end{block}

        \begin{block}{Introduction}
          \begin{itemize}
          \item some items and $\alpha=\gamma, \sum_{i}$
          \item some items
          \item some items
          \item some items
          \end{itemize}
          $$\alpha=\gamma, \sum_{i}$$
        \end{block}
      \end{column}
    \end{columns}
  \end{frame}
\end{document}
