\documentclass[final]{beamer}
\mode<presentation> {\usetheme{Dreuw}}
\usepackage{amsmath,amsthm,amssymb,latexsym, graphicx}
\usepackage[english]{babel}
\usepackage[utf8]{inputenc}
\usepackage[orientation=portrait,size=a0,scale=1.4,debug]{beamerposter}

\title{Sensitivity of a density-dependent population matrix model}
\author{Andr\'e Chalom$^1$ \& Paulo In\'acio Prado$^2$}
\institute{$^1$ Programa Interunidades de Pós-Graduação em Bioinformática, 
 $^2$ Instituto de Biociências, Universidade de São Paulo, Brasil.}

  \begin{document}
  \begin{frame}{} 
    \vfill
    \begin{block}{\large Abstract}
In recent years, we are seeing the formulation and use of
elaborate and complex models in ecological studies.
The questions related to the efficient, systematic and error-proof
exploration of parameter spaces are of great importance to better understand,
estimate confidences and make use of the output from these models. 
In this work,
we investigate a structured population growth model
using Latin Hypercube Sampling (LHS) and Extended Fourier Amplitude Sensitivity
Testing (eFAST).
    \end{block}
    \vfill
    \begin{columns}[t]
      \begin{column}{.48\linewidth}
        \begin{block}{Introduction}
          \begin{itemize}
		  \item Density dependence occurs when the growth rate of a population is regulated by the population density.
          \item Density dependence is credited as being responsible for the low density of many tree species, and consequently for the high species diversity \cite{Janzen70, Connel71}.
          \item {\em Euterpe edulis} (palmito ju\c cara) is an over-exploited species, commercially exploited for its edible heart of palm \cite{Pollak95}.
          \end{itemize}
        \end{block}
    \vfill
        \begin{block}{Mathematical model}
          \begin{itemize}
          \item The density independent growth of the population is modeled by a Lefkovitch matrix with seven size classes:
\begin{equation}
A = \left[
\begin{array} {ccccccc}
s_1 \cdot \overline{g_1} &   0 &   0 &   0 &   0 &   0 & F_7 \\
s_1 \cdot g_1 & s_2 \cdot \overline{g_2} &   0 &   0 &   0 &   0 &   0 \\
0 & s_2 \cdot g_2 & s_3 \cdot \overline{g_3} &   0 &   0 &   0 &   0 \\
0 &   0 & s_3 \cdot g_3 & s_4 \cdot \overline{g_4} &   0 &   0 &   0 \\
0 &   0 &   0 & s_4 \cdot g_4 & s_5 \cdot \overline{g_5} &   0 &   0 \\
0 &   0 &   0 &   0 & s_5 \cdot g_5 & s_6 \cdot \overline{g_6} &   0 \\
0 &   0 &   0 &   0 &   0 & s_6 \cdot g_6 & s_7 
\end{array}
\right]
\label{LefMatrixReparam}
\end{equation}
	   	  \item $s_i$ and $g_i$ are the probabilities of survival and growth into the next class, and $F_i$ is the number of offspring per tree, for each class. $\overline{g_i} = 1-g_i$.
		  \item The density dependent model uses the same matrix, but replaces $g_1$ by:
\begin{equation}
g_1 = \frac{ g_m }{1+ a N_1} \exp \left(- \frac{z}{\rho} N_7 \right)
\label{G_1}
\end{equation}
		  \item $N_1$ and $N_7$: number of seedlings and adults; $g_m$ and $a$: maximum growth rate and the strength of reduction in $g_1$; $z$ and $\rho$: crown area and plot size.
          \item All data on the species was extracted from \cite{SilvaMatos99}.
          \end{itemize}
        \end{block}
		\begin{block}{Analyses}
		\begin{itemize}
		\item We have generated Latin Hypercubes consisting of all variables for each model, and corrected for correlations between variables \cite{Huntington98} and evaluated the model at each combination of parameters.
		\item The Partial Rank Correlation Coefficient measures the strength of the monotonic relation between each input parameter $x_i$ and the result, discounting the effects of the other parameters \cite{Marino08}.
		\item The FAST algorithm estimates the importance of each parameter $x_i$ (main effect) and all 
		combinations of	parameters involving $x_i$ (total effect) \cite{Marino08}.
		\end{itemize}
		\end{block}
		\begin{block}{Conclusions}
		\begin{itemize}
		\item Quantification of the uncertainty in the assymptotic growth rate and the stable
		population size related to the uncertainty in the model inputs.
		\item Common framework to investigate linear and non-linear matrix models,
		including parameters not directly present on the matrix.
		\item Quantification of the importance of non-linearities and interactions between
		the input parameters, and incorporation of previous knowledge about the system.
		\end{itemize}
		\end{block}
		\vfill
        \begin{block}{Bibliography}
		\bibliographystyle{plain}
		\scriptsize{\bibliography{chalom}}
        \end{block}
      \end{column}
      \begin{column}{.48\linewidth}
        \begin{block}{Uncertainty analysis}
		\begin{figure}
		\begin{center}
		\includegraphics[width=0.6\textwidth]{leslie-ECDF}
		\end{center}
		\caption{Empirical Cumulative Distribution Function of the (a) dominant eigenvalue for the transition matrix of the density independent model and (b) predicted population size for the density dependent model.}
		\end{figure}
        \end{block}
		\vfill
        \begin{block}{Sensitivity: PRCC}
		\begin{figure}
		\begin{center}
		\includegraphics[width=0.6\textwidth]{leslie-PRCC}
		\end{center}
		\caption{Partial Rank Correlation Coefficients for the density independent (a) and density dependent (b) models. The bars are confidence intervals, generated by bootstraping 1000 times}
		\end{figure}
        \end{block}
		\vfill
        \begin{block}{Sensitivity: eFAST}
		\begin{figure}
		\begin{center}
		\includegraphics[width=0.6\textwidth]{leslie-fast}
		\end{center}
		\caption{eFAST analysis for the density independent (a) and density dependent (b) models. The bars represent the first and total order estimates for the sensitivity of each parameter in the model output.}
		\end{figure}
        \end{block}
      \end{column}
    \end{columns}
  \end{frame}
\end{document}
