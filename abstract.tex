\begin{abstract}
There is a growing trend in the use of mathematical modeling tools in the
study of many areas of the biological sciences. 

The use of computer models in science is increasing, specially in fields where laboratory
experiments are too complex or too costly, like ecology.

Questions of efficient, systematic and error-proof exploration of parameter spaces are
are of great importance to better understand, estimate confidences and make use of the 
output from these models. 

We present a survey of the proposed methods to answer these questions, with emphasis
on the Latin Hypercube Sampling and focusing on quantitative analysis of the results. We also
compare analytical results for sensitivity and uncertainty, where relevant, to LHS results.

This document contains a revision about the state-of-art uncertainty and sensitivity analyses,
with a practical example of applying the described techniques to two models of structured population growth.

During the progress of this work, a package of \R functions was developed
to facilitate the real world use of the above theoretical tools, freely available at
http://cran.r-project.org/web/packages/pse.

	  \par
	  \vspace{1em}
	  \noindent\textbf{Keywords:} uncertainty analysis, sensitivity analysis, numerical modeling, likelihood
\end{abstract}
